\section{Discussion}
\subsection{Escape velocity}
The figures in \ref{results:escape-velocity} indicate that the planets escape velocity is between
8.6 and 8.7 AU/yr. The exact solution is presented in \ref{eq:escape}, showing that the numerically
obtained escape velocity is lower than the analytical solution. This is a reasonable result as the 
analytical solution is able to account for the infinity-case, while the numerical solution can only
approach infinity by going 'far enough'. The numerical result of $v_{esc} = 8.7 \tfrac{AU}{yr}$ can 
perhaps be concidered close enough, as for the entire simulation, the planets distance from the sun 
does not stop increasing. If the amount of years were to be increased by an order of magnitude, 
however, one would have to increase the initial velocity closer to the exact solution in order for 
the planet to look like it is escaping. Thus, if there was enough time, one could continue to  
increase the amount of simulation years and time steps until a sufficiently accurate result is
obtained. Hardware limitations may surface if the numbers become very large.


\subsection*{Class hierarchy}

The class structure used in this project is adapted from that written by Anders Hafreager. Adapting it to further generalization of the N-body problem is worth a bit of consideration. Primarily the class \lstinline{solarSystem} needs to be malleable  in the type of object it contains. That is to say; the \lstinline{celestialBody} class needs be expanded to be able to host arbitrary physical attributes as demanded by the acting potential. Secondly the class would need to take a pointer to a function (or perhaps a struct) or set of such potentials to facilitate any given set of interactions. More complex algorithms and considerations could be implemented if one were to consider "grouping" mechanics that would enable efficient iterations over truly large amounts of particles.
