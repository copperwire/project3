\section{Method}

\subsection{Velocity-Verlet method}
\subsubsection{General Verlet}
This is a popular algorithm for solving Newton's equations of motion, and is also used in 
molecular dynamics simulations and computer graphics. In this case the algorithm will be 
applied to the gravitational force between to bodies in a simulated solar system. 
The problem is described by Newton's second law of motion, which will be the example for outlining
the method.

\begin{equation*}
m\frac{d^2x}{dt^2}= F(x,t)
\end{equation*}

Rewriting in terms of two coupled differential equations gives:
\begin{align*}
\frac{dx}{dt} &= v(x,t) \\
\frac{dv}{dt} &= \frac{F(x,t)}{m} = a(x,t)
\end{align*}


Taylor expanding $x$ gives:
\begin{equation*}
x(t+h) = x(t) + hx^{(1)}(t) + \frac{h^2}{2}x^{{(2)}}(t) + O(h^3)
\end{equation*}

Adding the corresponding Taylor expansion for $x(t-h)$, and discretizing the expressions,
the following is obtained:

\begin{align*}
x(t_i \pm h) &= x_{i\pm 1} \\
x_i &= x(t_i) \\
x_{i+1} &= 2x_i - x_{i-1} + h^2 x_i^{{(2)}} + O(h^4)
\end{align*}

\subsubsection{Velocity-Verlet}
In the above section it is found that velocity is not directly included in the equation, since
the function $x_i^{{(2)}} = a(x,t)$ is supposed to be known. If one would like to keep an eye on, say,
conservation laws for energy, the velocity is needed to find kinetic energy. It can be computed
using:
\begin{equation*}
x_i^{(1)} = \frac{x_{i+1} - x_{i-1}}{2h} + O(h^2)
\end{equation*}
Note that the algorithm for position depends on $x_{i-1}$, which is rather ficticious for $i=0$.
This is where velocity Verlet comes in. 
\newline
Taylor expanding the velocity gives:
\begin{equation*}
v_{i+1} = v_i + hv_i^{(1)} + \frac{h^2}{2}v_i^{{(2)}} + O(h^3)
\end{equation*}
Through Newton's second law an analytical expression for the derivative of the velocity can be
obtained:
\begin{equation*}
v_i^{(1)} = \frac{d^2x}{dt^2}|_i =\frac{F(x_i,t_i)}{m}
\end{equation*}
Adding the corresponding Taylor expansion for the derivative of the velocity:
\begin{equation*}
v_{i+1}^{(1)} = v_i^{(1)} + hv_i^{{(2)}} + O(h^2)
\end{equation*}
Since the error goes as $O(h^3)$ only terms up to the second derivative of the velocity are retained,
to reach the following expression:
\begin{equation*}
hv_i^{{(2)}} \approx v_{i+1}^{(1)} - v_i^{(1)}
\end{equation*}
This allows for rewriting the Taylor expansion for the velocity as
\begin{equation*}
v_{i+1} = v_i + \frac{h}{2}\big(v_{i+1}^{(1)} + v_i^{(1)}\big) + O(h^3)
\end{equation*}

The final expressions for position and velocity become:
\begin{align*}
x_{i+1} &= x_i + hv_i + \frac{h^2}{2}v_i^{(1)} + O(h^3) \\
v_{i+1} &= v_i + \frac{h}{2}\big(v_{i+1}^{(1)} + v_i^{(1)}\big) + O(h^3)
\end{align*}
It is important when implementing this method that the term $v_{i+1}^{(1)}$ depends on the position
$x_{i+1}$, which means the position must be calculated at $t_{i+1}$ before the next velocity can
be computed. Additionally, the derivative of the velocity at the time $t_i$ used when calculating
the updated position can be reused in the velocity update.




