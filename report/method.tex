\section{Method}

\subsection{Unit tests}
Unit tests are implemented to check that the numerical solutions are stable. More precisely, it is tested that the kinetic energy, potential energy and angular momentum all are conserved in the motion of the particles in the system. See section \ref{sec:cons} for justification that these should be conserved. 

\lstset{language=C++,
                basicstyle=\ttfamily,
                keywordstyle=\color{blue}\ttfamily,
                stringstyle=\color{red}\ttfamily,
                commentstyle=\color{green}\ttfamily,
                morecomment=[l][\color{magenta}]{\#}
                }
\subsection{Forces between bodies}
To find the motion of the planets, it is needed to find the forces between the planets. The following code block is used to calculate the forces between the objects in both the Euler and the Verlet method. Iterating over \lstinline{j = i+1} inside the loop over i ensures that all interactions are covered and only computet once. 
\begin{lstlisting}[frame=single]
  
for(int i=0; i<numberOfBodies(); i++) {

        CelestialBody &body1 = m_bodies[i];
        for(int j=i+1; j<numberOfBodies(); j++) {
            CelestialBody &body2 = m_bodies[j];
            vec3 deltaRVector = body1.position 
            - body2.position;
            double dr = deltaRVector.length();

            vec3 force_val = G* body1.mass*body2.mass
            *deltaRVector/(dr*dr*dr);
            body2.force += force_val;
            body1.force -= force_val;

            m_potentialEnergy += G* body1.mass
            *body2.mass/(dr);
        }
\end{lstlisting}

\subsection{Implementation Euler method}
The following code block is the implementation of the differential equations that are to be solved in this problem, using the Euler method. 

\begin{lstlisting}[frame=single]
    for(CelestialBody &body : system.bodies()){
        body.position += body.velocity * time_step;
        body.velocity += (body.force/body.mass)
        *time_step;
	} 
\end{lstlisting}

As seen from the code block, the number of FLOPS in the Euler method $5N$, where $N$ is the number of iterations.  

\subsection{Implementation of Verlet method}
The following code block is the implementation of the differential equations that are to be solved in this problem, using the Verlet method. 


\begin{lstlisting}[frame=single]

  for(CelestialBody &body : system.bodies()){
        vec3 prevAccel = body.force/body.mass;
        body.position += time_step*body.velocity
         + stepstep/2 * prevAccel;
        system.calculateForcesAndEnergy();
        body.velocity += time_step/2. 
        * (body.force/body.mass + prevAccel);
    }
\end{lstlisting}
As seen from the code block, the number of FLOPS in the Verlet method is $10N$, where $N$ is the number of iterations. 