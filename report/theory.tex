\section{Theory}
\subsection{Velocity-Verlet method}
\subsubsection{General Verlet}
This is a popular algorithm for solving Newton's equations of motion, and is also used in  
molecular dynamics simulations and computer graphics. In this case the algorithm will be  
applied to the gravitational force between to bodies in a simulated solar system. 
The problem is described by Newton's second law of motion, which will be the example for outlining
the method.

\begin{equation*}
m\frac{d^2x}{dt^2}= F(x,t)
\end{equation*}
Rewriting in terms of two coupled differential equations gives:
\begin{align*}
\frac{dx}{dt} &= v(x,t) \\
\frac{dv}{dt} &= \frac{F(x,t)}{m} = a(x,t)
\end{align*}
Taylor expanding $x$ gives:
\begin{equation*}
x(t+h) = x(t) + hx^{(1)}(t) + \frac{h^2}{2}x^{{(2)}}(t) + O(h^3)
\end{equation*}
Adding the corresponding Taylor expansion for $x(t-h)$, and discretizing the expressions,
the following is obtained:
\begin{align*}
x(t_i \pm h) &= x_{i\pm 1} \\
x_i &= x(t_i) \\
x_{i+1} &= 2x_i - x_{i-1} + h^2 x_i^{{(2)}} + O(h^4)
\end{align*}
\subsubsection{Velocity-Verlet}
In the above section it is found that velocity is not directly included in the equation, since
the function $x_i^{{(2)}} = a(x,t)$ is supposed to be known. If one would like to keep an eye on, say,
conservation laws for energy, the velocity is needed to find kinetic energy. It can be computed
using:
\begin{equation*}
x_i^{(1)} = \frac{x_{i+1} - x_{i-1}}{2h} + O(h^2)
\end{equation*}
Note that the algorithm for position depends on $x_{i-1}$, which is rather ficticious for $i=0$.
This is where velocity Verlet comes in.  
\newline 
Taylor expanding the velocity gives:
\begin{equation*}
v_{i+1} = v_i + hv_i^{(1)} + \frac{h^2}{2}v_i^{{(2)}} + O(h^3)
\end{equation*}
Through Newton's second law an analytical expression for the derivative of the velocity can be
obtained:
\begin{equation*}
v_i^{(1)} = \frac{d^2x}{dt^2}|_i =\frac{F(x_i,t_i)}{m}
\end{equation*}
Adding the corresponding Taylor expansion for the derivative of the velocity:
\begin{equation*}
v_{i+1}^{(1)} = v_i^{(1)} + hv_i^{{(2)}} + O(h^2)
\end{equation*}
Since the error goes as $O(h^3)$ only terms up to the second derivative of the velocity are retained,
to reach the following expression:
\begin{equation*}
hv_i^{{(2)}} \approx v_{i+1}^{(1)} - v_i^{(1)}
\end{equation*}
This allows for rewriting the Taylor expansion for the velocity as
\begin{equation*}
v_{i+1} = v_i + \frac{h}{2}\big(v_{i+1}^{(1)} + v_i^{(1)}\big) + O(h^3)
\end{equation*}
The final expressions for position and velocity become:
\begin{align*}
x_{i+1} &= x_i + hv_i + \frac{h^2}{2}v_i^{(1)} + O(h^3) \\
v_{i+1} &= v_i + \frac{h}{2}\big(v_{i+1}^{(1)} + v_i^{(1)}\big) + O(h^3)
\end{align*}
It is important when implementing this method that the term $v_{i+1}^{(1)}$ depends on the position
$x_{i+1}$, which means the position must be calculated at $t_{i+1}$ before the next velocity can
be computed. Additionally, the derivative of the velocity at the time $t_i$ used when calculating
the updated position can be reused in the velocity update.


\subsection{Two-body problem}

Initially in this project, the two-body problem is solved, with Earth orbiting the Sun. Actually the problem of Earth orbiting the Sun can be approximated as a one-body problem, as the mass of the Sun is a lot bigger than the mass of Earth, and so we can view it as Earth orbiting a stationary Sun. 

A planet orbiting a star, generally has an elliptical orbit, here we want to look at the case where the planet has an elliptical orbit. For this to be the case, the velocity of the planet must be perpendicular to the acceleration. 



\subsection{n-body problem}
The one-body problem can be solved analytically, and so can the two-body problem by splitting it up to two one-body problems. The three-body problem, however, can not be solved analytically. More generally, the n-body problem with $n>2$ cannot be solved analytically, and we depend on numerical tools to solve them. 

\subsection{Conservation laws}\label{sec:cons}

The angular momentum of a system is constant, unless there is a torque acting on the system. The torque is defined as 
\begin{equation}
\mathbf{\tau} = \mathbf{r}  \times \mathbf{F}
\end{equation}
In this case, $\mathbf{r}$ and $\mathbf{F}$ are parallel, so $\mathbf{\tau} = 0$
As there is no torque acting in the system, the angular momentum of the particles should be conserved. 

Also, both the kinetic energy and the potential energy should be conserved. 
From Newton's third law, we know that the action on one particle on the other is met by an equal and opposite reaction. Thus:
\begin{equation}
\sum_i \mathbf{F}_i  = 0
\end{equation}
We know that 
\begin{equation}
\sum_i \mathbf{F}_i = \sum\frac{d\mathbf{p}_i}{dt}
\end{equation}
And so, the sum of the total momentum must be a constant. For non-relativistic particles $\mathbf{p}= m\mathbf{v}$, and so, also the total velocity must be a constant. As the kinetic energy, $K = \frac{1}{2}mv^2$, this means that the total kinetic energy of the system is conserved. The total energy must be conserved, and so also the total potential energy of the system must be conserved. 
\subsection{Escape velocity}
Concidering a planet with an initial distance from the sun of 1 AU, the escape velocity can be
found using conservation of energy (assuming the planet - sun system can be treated as isolated).
Define kinetic energy as
$$E_k = \frac{1}{2}m_{p}v_{p}^2$$
and potential energy in the gravitational field as
$$E_p = -\frac{Gm_{sun}m_{p}}{r}$$
where $r$ is the distance between the center of mass of the planet and the sun, and $m_p$ is the
mass of the planet. Define also the potential
energy to be 0 when the planet is at an infinite distance away from the sun. The escape velocity can
then be seen as the initial velocity required for the planet to continue increasing its distance from
the sun until it comes to a stop when the distance reaches infinity. This gives the following
equation:
\begin{equation*}
E_{k,1} + E_{p,1} = E_{k,2} + E_{p,2}
\end{equation*}
where the left-hand side is the initial case and the right-hand side is the infinite-distance
case. Inserting the expressions for kinetic and potential energy, and the conditions for the
infinite-distance energies, gives:
\begin{align*}
\tfrac{1}{2}m_{p}v_{esc}^2 &- \frac{Gm_{sun}m_{p}}{r} = 0 \\
\tfrac{1}{2}m_{p}v_{esc}^2 &= \frac{Gm_{sun}m_{p}}{r} \\
v_{escape} &= \sqrt{\tfrac{2Gm_{sun}}{r}}
\end{align*}
Inserting $r = 1AU$ and $m_{sun} = 1$ gives 
\begin{equation}\label{eq:escape}
v_{esc} = \sqrt{2G} \approx 8.88\tfrac{AU}{yr}
\end{equation}
where the converted gravitational constant is $G = 39.42 \frac{(AU)^3}{M_{sun}(yr)^2}$

