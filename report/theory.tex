\section{Theory}
\subsection{Escape velocity}
Concidering a planet with an initial distance from the sun of 1 AU, the escape velocity can be
found using conservation of energy (assuming the planet - sun system can be treated as isolated).
Define kinetic energy as
$$E_k = \frac{1}{2}m_{p}v_{p}^2$$
and potential energy in the gravitational field as
$$E_p = -\frac{Gm_{sun}m_{p}}{r}$$
where $r$ is the distance between the center of mass of the planet and the sun, and $m_p$ is the
mass of the planet. Define also the potential
energy to be 0 when the planet is at an infinite distance away from the sun. The escape velocity can
then be seen as the initial velocity required for the planet to continue increasing its distance from
the sun until it comes to a stop when the distance reaches infinity. This gives the following
equation:
\begin{equation*}
E_{k,1} + E_{p,1} = E_{k,2} + E_{p,2}
\end{equation*}
where the left-hand side is the initial case and the right-hand side is the infinite-distance
case. Inserting the expressions for kinetic and potential energy, and the conditions for the
infinite-distance energies, gives:
\begin{align*}
\tfrac{1}{2}m_{p}v_{esc}^2 &- \frac{Gm_{sun}m_{p}}{r} = 0 \\
\tfrac{1}{2}m_{p}v_{esc}^2 &= \frac{Gm_{sun}m_{p}}{r} \\
v_{escape} &= \sqrt{\tfrac{2Gm_{sun}}{r}}
\end{align*}
Inserting $r = 1AU$ and $m_{sun} = 1$ gives 
\begin{equation}
v_{esc} = \sqrt{2G} \approx 8.88\tfrac{AU}{yr}
\end{equation}
where the converted gravitational constant is $G = 39.42 \frac{(AU)^3}{M_{sun}(yr)^2}$

